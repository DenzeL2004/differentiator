\documentclass[a4paper, 12pt] {article}
\usepackage{graphicx}
\usepackage[utf8]{inputenc}
\usepackage{multirow}
\usepackage{mathtext}
\usepackage[T2A]{fontenc}
\usepackage{titlesec}
\usepackage{float}
\usepackage{empheq}
\usepackage{amsfonts}
\usepackage{amsmath}
\title{Семестровая работа по математическому анализу}
\author{Даниил Коломийцев \thanks{спонсор В.В.Редкозубов}}
\date{November 2022}
\begin{document}
\maketitle
\newpage

\textbf{Пред нами предстоит задача полностью расшарить данное выражение:}

\begin{math}
 \cos (x)
\end{math}
\\ 
\\ 
\textbf{Возьмем 6-ую производную по аргументу 'x' исходного выражения:}

\begin{math}
 \cos (x)
\end{math}
\\ 
\\ 
1-ая производная:

\begin{math}
(-1) \cdot  \sin (x) \cdot 1
\end{math}
\\ 
\\ 
Кок, кок, кок и все

\begin{math}
 \sin (x) \cdot (-1)
\end{math}
\\ 
\\ 
2-ая производная:

\begin{math}
 \cos (x) \cdot 1 \cdot (-1)+ \sin (x) \cdot 0
\end{math}
\\ 
\\ 
Следовательно

\begin{math}
 \cos (x) \cdot (-1)
\end{math}
\\ 
\\ 
3-ая производная:

\begin{math}
(-1) \cdot  \sin (x) \cdot 1 \cdot (-1)+ \cos (x) \cdot 0
\end{math}
\\ 
\\ 
Кок, кок, кок и все

\begin{math}
 \sin (x) \cdot (-1) \cdot (-1)
\end{math}
\\ 
\\ 
4-ая производная:

\begin{math}
( \cos (x) \cdot 1 \cdot (-1)+ \sin (x) \cdot 0) \cdot (-1)+ \sin (x) \cdot (-1) \cdot 0
\end{math}
\\ 
\\ 
Надеюсь, данный переход вас не сильно шокировал

\begin{math}
 \cos (x) \cdot (-1) \cdot (-1)
\end{math}
\\ 
\\ 
5-ая производная:

\begin{math}
((-1) \cdot  \sin (x) \cdot 1 \cdot (-1)+ \cos (x) \cdot 0) \cdot (-1)+ \cos (x) \cdot (-1) \cdot 0
\end{math}
\\ 
\\ 
Кок, кок, кок и все

\begin{math}
 \sin (x) \cdot (-1) \cdot (-1) \cdot (-1)
\end{math}
\\ 
\\ 
6-ая производная:

\begin{math}
(( \cos (x) \cdot 1 \cdot (-1)+ \sin (x) \cdot 0) \cdot (-1)+ \sin (x) \cdot (-1) \cdot 0) \cdot (-1)+ \sin (x) \cdot (-1) \cdot (-1) \cdot 0
\end{math}
\\ 
\\ 
Следовательно

\begin{math}
 \cos (x) \cdot (-1) \cdot (-1) \cdot (-1)
\end{math}
\\ 
\\ 
\textbf{Разложим до 6-ого члена ряда Тейлора выражение:}

\begin{math}
 \cos (x)
\end{math}
\\ 
\\ 
Разложение по переменой 'x':

\begin{math}
x^{2} \cdot (-0,5)+1+x^{4} \cdot 0,041666667+x^{6} \cdot (-0,0013888889)
\end{math}
\\ 
\\ 

\end{document}
