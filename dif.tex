\documentclass[a4paper, 12pt] {article}
\usepackage{graphicx}
\usepackage[utf8]{inputenc}
\usepackage{multirow}
\usepackage{mathtext}
\usepackage[T2A]{fontenc}
\usepackage{titlesec}
\usepackage{float}
\usepackage{empheq}
\usepackage{amsfonts}
\usepackage{amsmath}
\title{Семестровая работа по математическому анализу}
\author{Даниил Коломийцев \thanks{спонсор В.В.Редкозубов}}
\date{November 2022}
\begin{document} 
\maketitle
\newpage

Пред нами предстоит задача полностью расшарить данное выражение:

\begin{math}
 \frac { \sin (x)}{10} \cdot (x-2)^{2}
\end{math}
\\
\\
Возьмем 4-ую произадную по аргументу 'x' исходного выражения

\begin{math}
 \frac { \sin (x)}{10} \cdot (x-2)^{2}
\end{math}
\\
\\
1-ая производная:

\begin{math}
 \frac { \cos (x) \cdot 1 \cdot 10- \sin (x) \cdot 0}{10 \cdot 10} \cdot (x-2)^{2}+ \frac { \sin (x)}{10} \cdot 2 \cdot (x-2)^{1} \cdot (1-0)
\end{math}
\\
\\
Наеюсь, данный переход вас не сильно шокировал

\begin{math}
 \frac { \cos (x) \cdot 10}{100} \cdot (x-2)^{2}+ \frac { \sin (x)}{10} \cdot (x-2) \cdot 2
\end{math}
\\
\\
2-ая производная:

\begin{math}
 \frac {((-1) \cdot  \sin (x) \cdot 1 \cdot 10+ \cos (x) \cdot 0) \cdot 100- \cos (x) \cdot 10 \cdot 0}{100 \cdot 100} \cdot (x-2)^{2}+ \frac { \cos (x) \cdot 10}{100} \cdot 2 \cdot (x-2)^{1} \cdot (1-0)+ \frac { \cos (x) \cdot 1 \cdot 10- \sin (x) \cdot 0}{10 \cdot 10} \cdot (x-2) \cdot 2+ \frac { \sin (x)}{10} \cdot ((1-0) \cdot 2+(x-2) \cdot 0)
\end{math}
\\
\\
Наеюсь, данный переход вас не сильно шокировал

\begin{math}
 \frac { \sin (x) \cdot (-1) \cdot 10 \cdot 100}{10000} \cdot (x-2)^{2}+ \frac { \cos (x) \cdot 10}{100} \cdot (x-2) \cdot 2+ \frac { \cos (x) \cdot 10}{100} \cdot (x-2) \cdot 2+ \frac { \sin (x)}{10} \cdot 2
\end{math}
\\
\\
3-ая производная:

\begin{math}
 \frac {((( \cos (x) \cdot 1 \cdot (-1)+ \sin (x) \cdot 0) \cdot 10+ \sin (x) \cdot (-1) \cdot 0) \cdot 100+ \sin (x) \cdot (-1) \cdot 10 \cdot 0) \cdot 10000- \sin (x) \cdot (-1) \cdot 10 \cdot 100 \cdot 0}{10000 \cdot 10000} \cdot (x-2)^{2}+ \frac { \sin (x) \cdot (-1) \cdot 10 \cdot 100}{10000} \cdot 2 \cdot (x-2)^{1} \cdot (1-0)+ \frac {((-1) \cdot  \sin (x) \cdot 1 \cdot 10+ \cos (x) \cdot 0) \cdot 100- \cos (x) \cdot 10 \cdot 0}{100 \cdot 100} \cdot (x-2) \cdot 2+ \frac { \cos (x) \cdot 10}{100} \cdot ((1-0) \cdot 2+(x-2) \cdot 0)+ \frac {((-1) \cdot  \sin (x) \cdot 1 \cdot 10+ \cos (x) \cdot 0) \cdot 100- \cos (x) \cdot 10 \cdot 0}{100 \cdot 100} \cdot (x-2) \cdot 2+ \frac { \cos (x) \cdot 10}{100} \cdot ((1-0) \cdot 2+(x-2) \cdot 0)+ \frac { \cos (x) \cdot 1 \cdot 10- \sin (x) \cdot 0}{10 \cdot 10} \cdot 2+ \frac { \sin (x)}{10} \cdot 0
\end{math}
\\
\\
Наеюсь, данный переход вас не сильно шокировал

\begin{math}
 \frac { \cos (x) \cdot (-1) \cdot 10 \cdot 100 \cdot 10000}{1e+08} \cdot (x-2)^{2}+ \frac { \sin (x) \cdot (-1) \cdot 10 \cdot 100}{10000} \cdot (x-2) \cdot 2+ \frac { \sin (x) \cdot (-1) \cdot 10 \cdot 100}{10000} \cdot (x-2) \cdot 2+ \frac { \cos (x) \cdot 10}{100} \cdot 2+ \frac { \sin (x) \cdot (-1) \cdot 10 \cdot 100}{10000} \cdot (x-2) \cdot 2+ \frac { \cos (x) \cdot 10}{100} \cdot 2+ \frac { \cos (x) \cdot 10}{100} \cdot 2
\end{math}
\\
\\
4-ая производная:

\begin{math}
 \frac {(((((-1) \cdot  \sin (x) \cdot 1 \cdot (-1)+ \cos (x) \cdot 0) \cdot 10+ \cos (x) \cdot (-1) \cdot 0) \cdot 100+ \cos (x) \cdot (-1) \cdot 10 \cdot 0) \cdot 10000+ \cos (x) \cdot (-1) \cdot 10 \cdot 100 \cdot 0) \cdot 1e+08- \cos (x) \cdot (-1) \cdot 10 \cdot 100 \cdot 10000 \cdot 0}{1e+08 \cdot 1e+08} \cdot (x-2)^{2}+ \frac { \cos (x) \cdot (-1) \cdot 10 \cdot 100 \cdot 10000}{1e+08} \cdot 2 \cdot (x-2)^{1} \cdot (1-0)+ \frac {((( \cos (x) \cdot 1 \cdot (-1)+ \sin (x) \cdot 0) \cdot 10+ \sin (x) \cdot (-1) \cdot 0) \cdot 100+ \sin (x) \cdot (-1) \cdot 10 \cdot 0) \cdot 10000- \sin (x) \cdot (-1) \cdot 10 \cdot 100 \cdot 0}{10000 \cdot 10000} \cdot (x-2) \cdot 2+ \frac { \sin (x) \cdot (-1) \cdot 10 \cdot 100}{10000} \cdot ((1-0) \cdot 2+(x-2) \cdot 0)+ \frac {((( \cos (x) \cdot 1 \cdot (-1)+ \sin (x) \cdot 0) \cdot 10+ \sin (x) \cdot (-1) \cdot 0) \cdot 100+ \sin (x) \cdot (-1) \cdot 10 \cdot 0) \cdot 10000- \sin (x) \cdot (-1) \cdot 10 \cdot 100 \cdot 0}{10000 \cdot 10000} \cdot (x-2) \cdot 2+ \frac { \sin (x) \cdot (-1) \cdot 10 \cdot 100}{10000} \cdot ((1-0) \cdot 2+(x-2) \cdot 0)+ \frac {((-1) \cdot  \sin (x) \cdot 1 \cdot 10+ \cos (x) \cdot 0) \cdot 100- \cos (x) \cdot 10 \cdot 0}{100 \cdot 100} \cdot 2+ \frac { \cos (x) \cdot 10}{100} \cdot 0+ \frac {((( \cos (x) \cdot 1 \cdot (-1)+ \sin (x) \cdot 0) \cdot 10+ \sin (x) \cdot (-1) \cdot 0) \cdot 100+ \sin (x) \cdot (-1) \cdot 10 \cdot 0) \cdot 10000- \sin (x) \cdot (-1) \cdot 10 \cdot 100 \cdot 0}{10000 \cdot 10000} \cdot (x-2) \cdot 2+ \frac { \sin (x) \cdot (-1) \cdot 10 \cdot 100}{10000} \cdot ((1-0) \cdot 2+(x-2) \cdot 0)+ \frac {((-1) \cdot  \sin (x) \cdot 1 \cdot 10+ \cos (x) \cdot 0) \cdot 100- \cos (x) \cdot 10 \cdot 0}{100 \cdot 100} \cdot 2+ \frac { \cos (x) \cdot 10}{100} \cdot 0+ \frac {((-1) \cdot  \sin (x) \cdot 1 \cdot 10+ \cos (x) \cdot 0) \cdot 100- \cos (x) \cdot 10 \cdot 0}{100 \cdot 100} \cdot 2+ \frac { \cos (x) \cdot 10}{100} \cdot 0
\end{math}
\\
\\
Наеюсь, данный переход вас не сильно шокировал

\begin{math}
 \frac { \sin (x) \cdot (-1) \cdot (-1) \cdot 10 \cdot 100 \cdot 10000 \cdot 1e+08}{1e+16} \cdot (x-2)^{2}+ \frac { \cos (x) \cdot (-1) \cdot 10 \cdot 100 \cdot 10000}{1e+08} \cdot (x-2) \cdot 2+ \frac { \cos (x) \cdot (-1) \cdot 10 \cdot 100 \cdot 10000}{1e+08} \cdot (x-2) \cdot 2+ \frac { \sin (x) \cdot (-1) \cdot 10 \cdot 100}{10000} \cdot 2+ \frac { \cos (x) \cdot (-1) \cdot 10 \cdot 100 \cdot 10000}{1e+08} \cdot (x-2) \cdot 2+ \frac { \sin (x) \cdot (-1) \cdot 10 \cdot 100}{10000} \cdot 2+ \frac { \sin (x) \cdot (-1) \cdot 10 \cdot 100}{10000} \cdot 2+ \frac { \cos (x) \cdot (-1) \cdot 10 \cdot 100 \cdot 10000}{1e+08} \cdot (x-2) \cdot 2+ \frac { \sin (x) \cdot (-1) \cdot 10 \cdot 100}{10000} \cdot 2+ \frac { \sin (x) \cdot (-1) \cdot 10 \cdot 100}{10000} \cdot 2+ \frac { \sin (x) \cdot (-1) \cdot 10 \cdot 100}{10000} \cdot 2
\end{math}
\\
\\

\end{document}
